\chapter{Conclusion}
In this thesis, we described and built an inexpensive, portable yet reasonably accurate sound localization system with two microphone arrays. We have analyzed different array architectures and showed that the localization accuracy varies with the array geometry and the location of the sound source (in terms of both the distance and the angle). In particular, regions close to the line connecting two microphones are generally more difficult to localize than regions close the line bisecting two microphones. This problem can be alleviated by employing multiple microphones that are placed at large distance from each other, but such arrangement affects the portability of the system. As a trade off between accuracy and portability, we demonstrated that a two array architecture achieves better accuracy compared to a single array system of similar size. By using cross-correlation output as the likelihood for different arrival time differences, each array generates a likelihood map for each possible location in the area. We demonstrated that by merging likelihood maps from the two arrays, we were able to achieve good localization accuracy (less than $3$ cm average error) in a local one meter by one meter region. Since two arrays generate their likelihood maps individually, this system does not require synchronized clock between the arrays, which made the overall system easier to design and more portable.

This system we have built can be installed in HCI applications that incorporate sound position and movement information. One can use this system to build virtual drawing applications where the user can draw with music, without physically touching the computer. One can also use this system to design interactive AI games. For example, a chess game with physical pieces can be developed where each piece is equipped with a motor and a music tag. Since the computer knows where all the pieces are, it knows which piece the user has moved and can make its corresponding move. A similar example is AI toy car racing game, where the player controls one car and the computer controls another car. With real time location information, the computer can control one car to compete with the player on a racing track. This system can also be used to in Augmented Reality (AR) applications. For example, users can wear a music tag on the finger, and then the system would be able to track finger movement. This particular setup can be used as a virtual mouse for wearable technologies such as Google Glass. 

For future directions, one can extend the system with another microphone on each array to perform 3D localization. However, the number of grid points increases exponentially with the number of dimensions, which translates into exponential increase in computation time. To keep the localization in real time (time lag of less than $0.1$ seconds), it is important to research more efficient ways of finding the intersection of hyperbolic cones the in the 3D space. Since the likelihood calculation on each grid point is simply a look up on the cross-correlation output from each microphone pair, the computation at each grid point is independent of each other. Therefore, one straightforward way to speedup calculation is to employ hardware accelerated parallel computing technologies such as GPU or FPGA. 

Another direction is to investigate multi-source localization. One approach is to allocate different frequency bands to different sound sources. However, the number of frequency bands are limited in the audible spectrum, and our results showed that TDOA estimation is more accurate for a wide band signal compared to a narrow band signal. To allow signals to occupy the entire frequency band, one potential approach is to devise protocols that allow time division multiplex of source signals, where each source signal occupies a different time slot. Another approach is for each source to have a different audio signature. If the localization module knows all sources' audio signatures, it can use that information to recover each source signal from the received superimposed waveform.
