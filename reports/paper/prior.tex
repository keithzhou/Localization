\chapter{Background}
Acoustic localization has been researched extensively in the literature. Localization techniques can be broadly categorized into Interaural Level Difference (ILD), Location Template Matching (LTM), and Time Difference of Arrival (TDOA) based approaches.

\section{ILD}
ILD techniques rely on the observation that signal intensity decays as the distance to the microhone increases. A microphone closer to the signal source would receive the signal with higher intensity than a microphone farther away. With multiple microphones, it is possible to infer the source location by comparing the signal intensity received at different microphones. Human's auditory system has used ILD cues to infer source direction~\cite{ild:2}, and this technique is most effective to localize high frquency sources, because they don't diffract effectively around the listener's head and produce a significant intensity difference.

Let $s(t)$ denote the audio signal. The received signal $\bar s(t) $ at microphone $i$ can be modeled as:
\[
\bar s_i(t) = \frac{s(t)}{d_i} + \delta_i(t)
\]
where $d_i$ is the distance between the audio source and microphone $i$, and $\delta_i(t)$ represents Gaussian noise.
The energy received at microphone $i$ is:
\begin{eqnarray}
E_i & = & \int \bar s_i(t)^2 dt \\
\label{eqn:ild1}
& = & \frac{1}{d_i^2} E_s + E_{\sigma_i}
\end{eqnarray}
It shows that the energy declays with the square of the distance. With two microphones, not considering the noise $\delta_i(t)$, energy received at both microphones has to satisfy equation~\ref{eqn:ild1}:
\begin{eqnarray}
E_1d_1^2 &=& E_2d_2^2\\
\label{eqn:ild}
\frac{E_1}{E_2} &=& \frac{d_2^2}{d_1^2}
\end{eqnarray}
It can be shown that, on a 2D plan, points satisfy equation~\ref{eqn:ild} form a circle when $E_1 \ne E_2$ and form a line when $E_1 = E_2$~\cite{ild:1}. With two microphones all points on this curve generate the same energy ratio and can not be distinguised from each other. Multiple approaches have been investigated to eliminate this ambiguity. \cite{ild:1} employed multiple microphones and used the intersection of circles from each microphone pair to estimate the source location. \cite{ild:2} combined ILD with Interaural Time Difference to estimate source direction (i.e azimuth).

ILD approach relies on the accurate measurement of the received energy ratio between microphone pairs. Any obstacle object between the sound source and any microphone would produce a significant distortion in the measured energy ratio. We find this approach to be too restrictive for our system, since in an interactive system we can not control whether or not the user places any obstacle between the sound source and the microphones.

\section{LTM}
In LTM based approaches, acoustic templates acquired from different locations are first stored in the system during a ``training'' phase. Localization can be performed by comparing the incoming waveform with the stored templates, and the location with the best matching template is chosen as the output. Different ways of extracting templates from raw acoustic source and different similarity measures have been investigated in the past. 

\cite{extended:tusi} and \cite{ltm:pham} investigated using max value from cross-correlation as a similarity measure to localize user taps on interactive surfaces. \cite{ltm:lpc} used L2 distance in the Linear Predictive Coding coefficient space as a similarity measure to localize taps on surfaces. \cite{ltm:tusi2} further explored accuracy improvement by using multiple templates for each location and speed improvement by merging multiple templates into one representative template.

The requirement of having a template for each location to be detected makes this approach too restrictive for our project, since we want the localization to be continuous in a 2D region.  Moreover, the need to recalibrate all locations during setup is too cumbersome for the end users in a portable system. Therefore, our main focus will be on TDOA based approaches.

\section{TDOA}
TDOA approaches exploit the difference of arrival time between the acoustic source and two fixed microphones on the plane. It can be easily shown that the acoustic sources with the same TDOA to two fixed microphones on the plane form a hyperbola. When you have more than two microphones, each pair would give a different hyperbola. The intersection of all the hyperbolas marks the source location. TDOA approaches rely on accurate estimates of arrival time differences between microphones. 

In \cite{tdoa:ppp}, authors used eight microphones mounted on the corners of a ping pong table to localize points where the ball hits the table. They used a threshold to determine the arrival time of acoustic signal. This approach works well in noise free environment but the performance degrades with background noise. Their approach also suffers from dispersive deflections that arrive before the main wavefront of the acoustic signal. To make it more robust, authors in \cite{tdoa:mit3} and \cite{tdoa:mit4} extracted descriptive parameters for each significant peak(e.g., peak height, width, mean arrival time). The algorithm then used extracted parameters to predict arrival time with a second order polynomial, the parameters of which were fitted during calibration at fixed locations.

Cross-correlation has also been used to measure signal arrival time differences\cite{tdoa:mit2, tdoa:micloc, tdoa:3}. Cross-correlation with prefiltering is known as \emph{generalized cross correlation (GCC)}. Different prefiltering approaches have been investigated to improve arrival time difference estimation~\cite{tdoa:gcc1,tdoa:gcc2,tdoa:gcc3}.

Under the GCC framework, the arrival time difference $t_0$ between two signals $x_1(t)$ and $x_2(t)$ can be estimated as:
\begin{eqnarray} \label{eq:gcc}
t_0 &=& \arg\max_{\tau} R_{x_1x_2}(\tau) \\\label{eq:gcc2}
R_{x_1x_2}(\tau) &=& \int_{-\infty}^\infty W(\omega) X_1(\omega) X_2^{*}(\omega) e^{j\omega\tau} d\omega
\end{eqnarray}
where $X_1(\omega)$ and $X_2(\omega)$ are Fourier Transforms of $x_1(t)$ and $x_2(t)$. $W(\omega)$ provides a way to prefilter signals passed to the cross correlation estimator. We focused on three ways of prefiltering the signal:
\begin{description}%[\IEEEsetlabelwidth{Very very long label}\IEEEusemathlabelsep]
\item[GCC] $W(\omega) = 1$. No prefiltering is done. This is unfiltered normal cross correlation.
\item[GCC\_PHAT] $W(\omega) = \frac{1}{\left|X_1(\omega)X_2^{*}(\omega)\right|}$. Each frequency is divided by its magnitude. Only phase information contributes to delay estimation.
\item[GCC\_PHAT\_SQRT] $W(\omega) = \frac{1}{\left|X_1(\omega)X_2^*(\omega)\right|^{0.5}}$. This is somewhere between GCC and GCC\_PHAT. Part of magnitude information is included in arrival time difference estimation.
\end{description}

To see the reasoning behind different prefitering approaches, we can seperate the magnitude and phase part of $X_1(\omega)$ and $X_2(\omega)$ in Equation~\ref{eq:gcc2}:
\begin{eqnarray}
\label{eqn:phat1}
R_{x_1x_2}(\tau) &=& \int_{-\infty}^\infty W(\omega) |X_1(\omega)||X_2(\omega)| e^{j(\omega\tau - (\angle{X_2(\omega)} - \angle{X_1(\omega)}) } d\omega\\
&=& \int_{-\infty}^\infty W(w)|X_1(\omega)| |X_2(\omega)| \cos(\omega\tau - (\angle X_2(\omega) - \angle X_1(\omega))) d\omega
\end{eqnarray}
We look at the real part of equation~\ref{eqn:phat1} only since both $x_1(t)$ and $x_2(t)$ are real valued signals. When $\dot \tau$ is the true arrive time difference between $x_1(t)$ and $x_2(t)$, $\omega \dot \tau - (\angle X_2(\omega) - \angle X_1(\omega)) = 0$, and $\cos(\omega \dot \tau - (\angle X_2(\omega) - \angle X_1(\omega))) = 1$. Therefore, $\cos(\omega\tau - (\angle X_2(\omega) - \angle X_1(\omega)))$ can be seen as a measure of the phase error, and $W(w)|X_1(\omega)||X_2(\omega)|$ describes how the error should be weighted at each frequency. The TDOA estimator essentially sums the weighted phase error at each frequncy. 

Without any prefitering (i.e $W(\omega)=1$), the estimator weighs the phase error at each frequency by the magnitude of the signal at that frequency. In this weighting scheme, phase error at frequencies with higher magnitudes are penalized more compared to frequencies with a lower magnitude. This weighting is approporiate if there is only one source present, since freqeuncies with higher intensities have higher Signal to Noise Ratio (SNR). It makes sense to place higher weights at frequencies with higher SNR. 

However, with multiple sources, the source with the highest volume will dominate the phase error estimation, but there is no particular reson to assign higher weights to signals with higher volume. All sources should contribute equally in phase error estimation. In GCC\_PHAT,  $W(\omega)$ is set to $\frac{1}{|X_1(\omega)||X_2(\omega)|}$. In effect it weighs phase error uniformly across freqiency. 
