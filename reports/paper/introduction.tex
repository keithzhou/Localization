%paper \section{Introduction}
\chapter{Introduction}

Accurate indoor localization allows creation of novel applications with surrounding awareness that uses position and movement information as input. One application is to allow users to draw with music without physically touching the computer. Another example is to build AI games with physical pieces such as toy car racing where the computer controls some of the toy cars. In this work, we aim to build such a source localization system that is portable, inexpensive, yet reasonably accurate for localization in a small area. 

Global Positioning System (GPS) is the prevailing technology used for outdoor localization. Commercial grade GPS has an average error of a few meters, depending on the size and quality of the receiver~\cite{intro:gps}. While accuracy in this range is good for many applications including driving navigation and vehicle tracking, it does not provide enough precision for local movement tracking. Ultrasound based indoor localization approaches on the other hand, have achieved sub-centimeter accuracy~\cite{intro:ultra}. However, ultrasound systems require the use of expensive transducers.

Bluetooth and Wi-Fi based technologies have gained popularity in indoor positioning recently, mainly due to the widespread deployment of bluetooth tags and Wi-Fi stations in public spaces. In these systems, signal strength received from different base stations are used for the estimation of the device location. However, their reported accuracy are in the range of $1$ to $5$ meters~\cite{intro:blue, intro:loc}, which is not enough for local movement tracking.

In this project, we have built a localization system with reasonably high precision for small area using microphone arrays that localize typical audio sources. Our system is built with inexpensive electret microphones mounted on portable frames. Users can interact with our system using any device that has audio output such as a mobile phone.

In this paper, we first discussed in Section II some prior relevant research and approaches in sound localization. In Section III, we evaluated different array architectures and their impact on localization accuracy. We demonstrated that a two array system outperforms a single array system of similar physical dimension. In Section IV, we presented the chosen architecture along with hardware details. Finally, experiment details and results were presented in Section V.
