%paper \section{Introduction}
\chapter{Introduction}

Accurate indoor localization allows creation of novel applications with surrounding awareness that uses position and movement information as input. 

Global Positioning System (GPS) is the prevailing technology used for outdoor localization. A set of 32 satellites equipped with synchronized atomic clock orbit around the earth and broadcast their time and position information at fixed intervals. A GPS receiver listens to at least 4 satellites and uses the broadcasted timing information to infer the location. Commercial grade GPS has an average error of a few meters, depending on the size and quality of the receiver~\cite{intro:gps}. While accuracy in this range is good for many applications including driving navigation and vehicle tracking, it does not provide enough precision for local movement tracking. 

Ultrasound based indoor localization approaches, on the other hand, have achieved sub-centimeter accuracy~\cite{intro:ultra}. These systems employ multiple ultrasound receivers and use the arrival time difference between the sound source and receivers to infer the source location. Although obtaining good accuracy, ultrasound systems require the use of expensive transducers.

Bluetooth and Wi-Fi based technologies have gained popularity in indoor positioning recently, mainly due to the widespread deployment of bluetooth tags and Wi-Fi stations in public spaces. The fact that modern consumer devices such as mobile phones and tablets are commonly equipped with Wi-Fi and bluetooth modules also made this approach particular attractive because no hardware needs to be installed on the user end. Relevant commercial user case includes applications that serve advertisements and coupons to consumers in shopping centers based on the customer's location. In these systems, device signal strength received at different base stations are used for the estimation of the user device location. However, their reported accuracy are in the range of $1$ to $5$ meters~\cite{intro:blue, intro:loc}, which is not accurate enough for local movement tracking.

In this work, we have built a source localization system that is portable, inexpensive, yet reasonably accurate for localization in a small area. To test the accuracy of our localization system, we perform acoustic localization experiments in an one meter by one meter area. The experiment area size is adequate for our purposes because human arm movement will generally be constrained in this area. Our experiment show that our system localizes both static and moving acoustic source with good accuracy in this area. Our system is built with inexpensive electret microphones mounted on portable frames. Users can interact with our system using any device that has audio output such as a mobile phone.

This system can be used in Human-Computer Interaction (HCI) applications that incorporate sound position and movement information. One can use this system to build virtual drawing applications where the user can draw with music without physically touching the computer. One can also use this system to design interactive artificial intelligence (AI) games with audible physical game pieces. For example, a chess game with physical pieces can be developed where each piece is equipped with a motor and a music tag. Since the computer knows where all the pieces are and which one the user has moved, it can make its corresponding move. Another similar example is to build an AI toy car racing game, where the player controls one car and the computer controls another. With real time location information of both cars, the AI engine can compete with the player on a racing track. 

This system can also be used to in Augmented Reality (AR) applications. For example, the user can wear a music tag on one finger, and then the system would be able to track the user's finger movements. This particular setup can be used as a virtual mouse for wearable technologies such as Google Glass. 


In this thesis, we first discussed in Chapter II some prior relevant research and approaches in sound localization. In Chapter III, we evaluated different array architectures and their impact on localization accuracy. We demonstrated that a two array system outperforms a single array system of similar physical dimension. In Chapter IV, we first presented the chosen architecture along with hardware details, followed by experiment details and results. 
