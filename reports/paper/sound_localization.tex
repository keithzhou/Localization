\section{Sound Localization}
Constant difference of distance to two fixed points trace out a hyperbola in the 2D plane. Since each pair of microphones gives a hyperbola curve in the plane, localization becomes finding intersection of hyperbolas when more than two microphones are used. Accurate localization relies on accuracte estimate of delay differences between microphones.

Generalized Cross Correlation provides a framework to estimate delay differences $t_0$ between two signals $x_1(t)$ and $x_2(t)$:
\[
t_0 = \arg\max_{\tau} \int_{-\infty}^\infty W(\omega) X_1(\omega) X_2^{*}(\omega) e^{j\omega\tau} d\omega
\]
, where $X_1(\omega)$ and $X_2(\omega)$ are Fourier Transform of $x_1(t)$ and $x_2(t)$. $W(\omega)$ provides a way to prefilter the signals passed to cross correlation estimator. We experiment with three ways of prefiltering the signal:
\begin{description}[\IEEEsetlabelwidth{Very very long label}\IEEEusemathlabelsep]
\item[GCC] $W(\omega) = 1$. No prefiltering is done. This is normal cross correlation.
\item[GCC\_PHAT] $W(\omega) = \frac{1}{\left|X_1(\omega)X_2^{*}(\omega)\right|}$. Each frequency is divideded by its magnitude. Only phase information contributes to delay estimation
\item[GCC\_PHAT\_SQRT] $W(\omega) = \frac{1}{\left|X_1(\omega)X_2^*(\omega)\right|^{0.5}}$. This is somewhere between GCC and GCC\_PHAT. part of magnitude information is included in delay estimation.
\end{description}
